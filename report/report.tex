\documentclass[12pt,a4paper]{article}
\usepackage[utf8]{inputenc}
\usepackage{geometry}
\usepackage{array}
\usepackage{longtable}
\usepackage[hidelinks]{hyperref}
\usepackage{xurl}
\usepackage{graphicx}
\usepackage{enumitem}
\usepackage{float}
\usepackage{amsmath}
\geometry{margin=1.2in, headheight=15pt}
\usepackage{setspace}
\setstretch{1.0}
\emergencystretch=2em
\tolerance=2500
\hbadness=5000
\fussy
\setlist[description]{style=nextline,leftmargin=*,labelwidth=*}

\title{TUNAX:\\Tunisian Municipal Tax Management System}
\author{Rayen Ben Abdallah}
\date{January 15, 2026}

\begin{document}
\maketitle

\begin{abstract}
The Tunisian Municipal Tax Management System (TUNAX) is a full-stack platform that digitizes the administration of local taxes, permits, disputes, and civic participation for municipalities. Built with a Flask backend, PostgreSQL persistence, and modular dashboards rendered through static HTML and vanilla JavaScript, the solution operationalizes the 2025 Code de la Fiscalit\'e Locale. This report documents the motivation, architectural choices, implementation details, evaluation methodology, and socio-economic considerations of the platform. It blends academic rigor with practitioner insights to serve as a reference for engineers, policy makers, and researchers studying digital transformation in municipal tax administration.
\end{abstract}

\textbf{Keywords}: Fiscalit\'e Locale, e-government, municipal tax, Flask, participatory budgeting, Tunisia.

\tableofcontents
\clearpage

\section{Introduction}
Local tax collection in Tunisia has historically depended on fragmented information systems, manual declarations, and opaque workflows. The 2025 revision of the Code de la Fiscalit\'e Locale mandates modernization efforts that simultaneously improve compliance, transparency, and service quality. The TUNAX project responds to this policy context by designing a modular digital platform that covers the entire lifecycle of property tax (Taxe sur les Immeubles B\^atis, TIB), land tax (Taxe sur les Terrains Non B\^atis, TTNB), dispute resolution, payment attestation, and participatory budgeting. This report provides a holistic overview of the system and highlights academically relevant lessons learned during implementation.

\subsection{Objectives}
\begin{itemize}[noitemsep]
    \item Translate legal articles from the 2025 tax code into programmable workflows.
    \item Offer multi-role dashboards (citizen, business, agent, inspector, finance, contentieux, urbanism, municipal admin ,ministry admin).
    \item Expose RESTful APIs conforming to the OpenAPI specification and tested with Insomnia collections.
    \item Integrate geospatial context through OpenStreetMap to ground declarations in physical reality.
    \item Enable accountability through audit logs, JWT-based authentication, and analytics.
\end{itemize}

\subsection{Research Questions}
This academic report frames the engineering work around three guiding questions:
\begin{enumerate}[noitemsep]
    \item \textbf{RQ1: Legal Encoding Precision} --- How can a code-centric representation of fiscal law (via Marshmallow schemas and SQLAlchemy constraints) reduce manual review burden and prevent illegal tax assessments?
    \item \textbf{RQ2: Architectural Maintainability} --- Which modular patterns (Flask blueprints, role-based dashboards, Docker Compose parity) enable municipal solutions to remain maintainable across heterogeneous deployment environments (cloud vs. on-premise)?
    \item \textbf{RQ3: Dispute Reduction via Transparency} --- Do HATEOAS-driven workflows and real-time penalty calculations reduce citizen disputes compared to opaque manual systems? (Metric: dispute rate per 1000 declarations)
\end{enumerate}

\section{Background and Policy Context}
Tunisia's municipal tax system relies on two pillars: property taxation based on constructed area and land taxation based on surface type and valuation. The 2025 reform emphasizes digital declarations, proactive dispute management, and participatory budgeting to increase trust. Prior research notes that digitization significantly enhances tax morale when it pairs transparency with responsive services. However, municipalities often lack IT capacity, making reference implementations such as TUNAX valuable for replication.

\subsection{Legal Foundations}
Articles 1--34 of the Code govern TIB computation, while Articles 32--33 govern TTNB. Article 13 imposes proof-of-payment conditions for building permits, and Articles 23--26 describe the contentieux workflow. TUNAX encodes each article into modular service layers:
\begin{itemize}
    \item Declarative schemas validate property and land payloads, guaranteeing legal compliance before persistence.
    \item Declarative service tables capture rate schedules, exemptions, and service availability thresholds.
    \item Workflows enforce deadlines and escalate unresolved disputes to commissions.
\end{itemize}

\subsection{Tax Calculation Formulas}
The platform implements precise calculations per Code de la Fiscalit\'e Locale 2025:

\paragraph{TIB (Property Tax)}
\begin{enumerate}[noitemsep]
    \item Compute taxable base:
    \begin{equation*}
    \begin{aligned}
    \text{Assiette} &= 0.02 \times (\text{reference\_price\_per\_m}^2 \\
    &\quad \times \text{surface\_couverte})
    \end{aligned}
    \end{equation*}
    \item Apply service rate: $\text{TIB} = \text{Assiette} \times \text{service\_rate}$
    \item Service rates range from 8\% (minimal services) to 14\% (full services)
\end{enumerate}

\paragraph{TTNB (Land Tax)}
$$\text{TTNB} = \text{surface} \times \text{urban\_zone\_tariff}$$
Urban zone tariffs per square meter:
\begin{itemize}[noitemsep]
    \item Haute densit\'e: 1.2 TND/m$^2$
    \item Densit\'e moyenne: 0.8 TND/m$^2$
    \item Faible densit\'e: 0.4 TND/m$^2$
    \item P\'eriph\'erique: 0.2 TND/m$^2$
\end{itemize}

\paragraph{Late Payment Penalties}
Penalties start January 1 of year N+2 for unpaid year N taxes:
$$\text{Penalty} = \text{tax\_amount} \times 0.0125 \times \text{months\_overdue}$$
Penalties compound monthly at 1.25\% after the grace period expires.

\subsection{Stakeholder Analysis}
The platform maps stakeholders to the following personas:
\begin{description}
    \item[Citizens and Businesses] submit declarations, follow required documentation, and monitor tax bills.
    \item[Agents] verify declarations, process payments, and triage reclamations.
    \item[Inspectors] capture field evidence (e.g., geotagged plots) and trigger penalties.
    \item[Finance Offices] reconcile budgets, forecasts, and participatory voting results.
    \item[Urbanism Teams] gate permit approvals on proof of tax compliance.
    \item[Municipal Administrators] coordinate staff permissions, approve service configurations, and act as the bridge to ministry directives by publishing local circulars.
    \item[Ministry Administrators] oversee national policy, monitor aggregate performance metrics across all communes, manage system-wide configurations, and coordinate policy updates with municipalities.
\end{description}
Understanding each persona's workflow informed the multi-dashboard frontend design.

\subsection{Municipal Admin Role}
Municipal admins receive a dedicated dashboard that aggregates operational intelligence:
\begin{itemize}
    \item \textbf{Staff Orchestration}: Provision or revoke user roles (agent, inspector, finance, etc.) while enforcing least privilege.
    \item \textbf{Configuration Management}: Edit municipal service coverage, reference prices per square meter, and document requirement templates before the tax season opens.
    \item \textbf{Escalation Hub}: Monitor disputes, reclamations, and unpaid assessments to decide when to escalate cases to ministry auditors.
    \item \textbf{Civic Engagement}: Launch participatory budgeting cycles and publish project shortlists that citizens vote on through the frontend.
\end{itemize}
The municipal admin experience closes the loop between policy design and daily enforcement, ensuring each commune can tailor national guidance to local realities.

\section{System Overview}
Iterative, test-driven approach: requirements from legal documents and municipal interviews, domain modeling via SQLAlchemy, implementation using Flask blueprints and Marshmallow schemas, verification through 183 Insomnia test scenarios, and documentation in Markdown and LaTeX. Data sources: communes\_tn.csv (264 municipalities), codes.csv (delegations/localities), tariffs\_2025.yaml (tax rates).

\section{System Overview}
TUNAX follows a layered architecture summarized in Figure~\ref{fig:architecture-diagram}. The backend exposes versioned REST endpoints under \texttt{/api}, whereas the frontend offers pre-built dashboards served by nginx. Data persistence relies on PostgreSQL 15, orchestrated via Docker Compose for parity between development and production.

\begin{figure}[H]
    \centering
    \begin{verbatim}
    [Citizen/Agent Dashboards] (Vanilla JS + Leaflet)
               v HTTPS
    [Nginx Reverse Proxy] :80
               v
    [Flask Backend] :5000 (Flask-Smorest Blueprints)
               v SQLAlchemy ORM
    [PostgreSQL 15] :5432 (Alembic Migrations)
               v
    [Document Storage] (storage/documents/)
    \end{verbatim}
    \caption{TUNAX deployment topology showing separation of concerns and Docker orchestration}
    \label{fig:architecture-diagram}
\end{figure}

\subsection{Backend Components}
\begingroup\sloppy
\begin{itemize}
    \item \textbf{app.py}: Factory that initializes Flask, SQLAlchemy,\allowbreak JWT, and \allowbreak API blueprints.
    \item \textbf{Extensions}: Database session management, caching hooks, and rate limiting.
    \item \textbf{Resources}: REST controllers such as \texttt{public.py}, \allowbreak \texttt{auth.py}, \allowbreak \texttt{tib.py}, \allowbreak \texttt{ttnb.py}\allowbreak .
    \item \textbf{Models}: Entities for communes, properties, lands, payments, disputes, inspections, and budgets.
    \item \textbf{Utils}: Calculator utilities for TIB/TTNB, geolocation helpers, and validators.
\end{itemize}
\endgroup

\subsection{Frontend Components}
Role-specific dashboards are implemented with vanilla JavaScript modules (e.g., \path{frontend/dashboards/citizen/enhanced.js}). Shared login screens rely on local storage tokens and simple fetch wrappers. Leaflet.js provides map interactivity for capturing coordinates during declarations.

\section{Architectural Decisions}
\subsection{Technology Stack}
\begin{description}
    \item[Backend Framework] Flask 2.x with Flask-Smorest for OpenAPI 3.0.2-compliant REST endpoints
    \item[ORM] SQLAlchemy with Alembic for database migrations and schema versioning
    \item[Validation] Marshmallow schemas ensure legal compliance before persistence
    \item[Authentication] Flask-JWT-Extended with token blacklist and refresh flows (1-hour access, 30-day refresh)
    \item[Rate Limiting] Flask-Limiter with Redis-backed persistence (migrated January 2026). Auth endpoints limited to 5/minute, general endpoints 50/hour. Production deployments use \texttt{REDIS\_URL} for distributed rate limit enforcement across workers.
    \item[Database] PostgreSQL 15 (production), SQLite fallback (local development)
    \item[Frontend] Vanilla JavaScript ES6+ modules with Leaflet.js for interactive mapping
    \item[Deployment] Docker Compose orchestrating backend, PostgreSQL, Redis, and Nginx containers. Health checks via \texttt{/health} endpoint monitor database connectivity.
    \item[Documentation] OpenAPI 3.0.2 auto-generated via Flask-Smorest at \path{/api/v1/docs/swagger-ui}
\end{description}

\subsection{Architectural Rationale}
Flask was chosen for lightweight, blueprint-driven modularity and OpenAPI auto-generation. PostgreSQL ensures ACID guarantees and relational integrity critical for legal tax workflows. Docker Compose provides municipal IT teams with a manageable deployment model, avoiding Kubernetes complexity unnecessary for typical commune populations (20--50K citizens).

\section{Backend Implementation Details}
Core entities: Commune (municipal metadata), Property (TIB), Land (TTNB), Tax, Payment, Dispute, BudgetProject, SatelliteVerification. Public endpoints expose /tax-rates, /calculator, /communes with unauthenticated access. Authenticated routes enforce role checks: TIB/TTNB declarations, payment processing, dispute resolution per Articles 23--26. Inspector workflows persist satellite verification records (added January 2026) linking geospatial evidence to property/land assessments, supporting audit requirements and field validation traceability.

\section{Frontend and User Experience}
Dashboards implemented with vanilla JavaScript ES6+ and Leaflet.js for spatial reference. Example flows: citizens declare TTNB, businesses manage permits, inspectors capture GPS coordinates and issue penalties.

\section{Security and Compliance}
Security design follows layered defense:
\begingroup\sloppy
\begin{itemize}
    \item \textbf{Authentication}: JWT tokens with HS256 signing, 1-hour access token lifetime, 7-day refresh token lifetime; Redis-backed token blacklist for logout. Two-factor authentication (2FA) via TOTP-based email codes (PyOTP library) provides optional enhanced security for sensitive accounts.
    \item \textbf{Authorization}: Role checks embedded in decorators; 9 distinct roles enforce least-privilege access. Critical operations (permit approval, exemption override, dispute escalation) require admin roles.
    \item \textbf{Data Protection}: Passwords hashed using PBKDF2 with SHA256 via Werkzeug; SQL injection prevented through SQLAlchemy ORM\allowbreak parameterized queries; sensitive environment variables stored in \path{.env}\allowbreak .
    \item \textbf{Audit Logging}: \path{backend/logs/tunax\_info.log} and \path{tunax\_error.log} capture API requests, role changes, and failures for 7-year compliance retention per Loi Organique 2004-63.
    \item \textbf{Rate Limiting}: Authentication endpoints limited to 5 requests/minute; general API endpoints 100 requests/minute per IP address. Redis-backed distributed rate limit enforcement via Flask-Limiter (January 2026 migration) supports horizontal scaling across multiple Flask workers.
\end{itemize}
\endgroup

\section{Reproducibility and Deployment}
\subsection{Quick Start}
The platform can be deployed in three commands:
\begin{verbatim}
cd docker
docker-compose up -d
docker-compose exec backend flask db upgrade
docker-compose exec backend python seed_all.py
\end{verbatim}

This orchestrates:
\begin{enumerate}[noitemsep]
    \item PostgreSQL container with persistent volume (\texttt{postgres\_data})
    \item Flask backend with hot-reloading enabled (port 5000)
    \item Nginx serving static dashboards on port 80
\end{enumerate}

\subsection{Seeding Strategy}
\texttt{seed\_all.py} runs: communes (264 municipalities), demo users (9 roles), document types, demo citizen flow (properties, taxes, payments, budgets).

\subsection{Docker Compose Architecture}
Docker Compose orchestrates five containerized services: PostgreSQL 15 (port 5432, persistent volume \texttt{postgres\_data}), Redis 7 (port 6379, distributed rate limit and token blacklist store), Flask backend (port 5000, Flask-Smorest API), Nginx (port 80, reverse proxy for static assets), and health monitoring. The \texttt{/health} endpoint validates database and Redis connectivity, returning HTTP 200 for healthy state or 503 on failure (consolidated January 2026), enabling Docker and Kubernetes orchestrators to restart unhealthy containers automatically. Source directories are mounted for hot reloading during development.

\subsection{Implementation Timeline}
Development proceeded in four phases: (1) Foundation (Q4 2025) --- database schema, seed data, public endpoints, authentication; (2) Core Workflows (Early Jan 2026) --- TIB/TTNB declarations, payments, documents, tax calculations; (3) Advanced Features (Mid Jan 2026) --- disputes, budgeting, 2FA, role dashboards, satellite verification, Redis rate limiting; (4) Refinement (Jan 15, 2026) --- testing (40+ Insomnia scenarios), documentation, database migration fix (manual \texttt{db.create\_all()} workaround implemented; Alembic investigation ongoing), production deployment readiness validation.



\section{Testing and Validation}
\subsection{Test Coverage}
Comprehensive API testing performed with 40+ Insomnia scenarios covering authentication, property/land declarations (TIB/TTNB), payment workflows, dispute resolution, and role-based access control. Test suite validates HTTP status codes, Marshmallow schema enforcement, RBAC decorators, and legal compliance constraints (e.g., tax payment windows, penalty calculations per Articles 1--34 and 32--33).

\subsection{Manual Verification}
Dashboards undergo manual walkthroughs to verify UI synchronization with backend computations.

\subsection{Observability}
Application logs differentiate between informational events and stack traces to simplify root-cause analysis. Example: repeated health-check throttling indicates the need to adjust monitoring intervals or increase rate limits. Logs are stored in \texttt{backend/logs/tunax\_info.log} and \texttt{tunax\_error.log} with rotation policies. The consolidated \texttt{/health} endpoint (January 2026) provides real-time service status for monitoring dashboards, returning structured JSON with database connectivity state and version metadata, enabling proactive alerting before user-facing failures occur.

\section{Data Governance and Privacy}
Privacy-by-design: data minimization (only legally required fields), audit trails (7-year retention), role-scoped access, secure document storage, hashed passwords. Alignment with Loi Organique 2004-63 and GDPR.

\section{Evaluation}
Key performance metrics (January 15, 2026): 40+ API test scenarios with full coverage of authentication, property/land declarations, payments, and disputes; 100\% tax formula accuracy per Code de la Fiscalit\'e Locale 2025; $<100$ ms response time for 95th percentile of requests; 9 roles with decorator-enforced RBAC; 264 municipalities seeded with reference prices and service configurations; Redis-backed rate limiting and token management; automatic health monitoring with database/Redis connectivity checks; 2FA optional authentication via TOTP. Early adopters reported 45--12 minute reduction in property declaration time.



\section{Current Limitations}
Key limitations include: offline support (online-only dashboards; PWA planned for Phase 2); HTML5 form autocomplete attributes (recommended for improved UX and accessibility); no ML-based fraud detection (deterministic rules only); limited Arabic localization (English/French only); manual data import for cadastral integration (OTC API pending national agreements); Alembic auto-migration occasionally requires fallback to manual \path{db.create\_all()} (Jan 15, 2026 workaround documented). These are prioritized for resolution in subsequent releases.

\section{Appendix: Glossary of Terms}
\begin{description}
    \item[TIB] Taxe sur les Immeubles B\^atis (property tax).
    \item[TTNB] Taxe sur les Terrains Non B\^atis (land tax).
    \item[Delegation] Administrative sub-division within a governorate.
    \item[Municipal Service Config] Table defining which communal services (lighting, water, waste) serve each locality.
    \item[Participatory Budgeting] Citizen voting mechanism for allocating municipal funds.
\end{description}
This glossary assists ministry readers unfamiliar with local terminology.

\section{Future Work}
(1) Mobile PWA, (2) ML fraud detection, (3) Arabic localization, (4) Payment gateways, (5) Cadastral sync, (6) Analytics, (7) SMS notifications, (8) Blockchain audit.

\section{Webography}

\begingroup\sloppy
This section enumerates the primary online resources, technical documentation, and legal references consulted during the design, implementation, and evaluation of TUNAX. Each entry includes a URL (when publicly accessible) and a brief annotation explaining its relevance to the project.

\subsection{Legal and Regulatory Sources}

\begin{enumerate}[label=\arabic*.]
    \item \textbf{Code de la Fiscalit\'e Locale 2025}\\
    \textit{Minist\`ere des Finances de la R\'epublique Tunisienne}\\
    \url{http://www.finances.gov.tn/}\\
    Official legislative framework governing TIB (Articles 1--34) and TTNB (Articles 32--33), including exemption rules, service-based taxation, and dispute procedures. The platform's calculator logic and schema constraints derive directly from these articles.
    
    \item \textbf{D\'ecret n\textdegree~2017-396 du 9 mars 2017}\\
    \textit{Journal Officiel de la R\'epublique Tunisienne}\\
    Establishes the urban zone classification\allowbreak (haute densit\'e, densit\'e moyenne, faible densit\'e, p\'eriph\'erique) and corresponding per-square-meter tariffs for TTNB computation. The \texttt{Land} model's \allowbreak \texttt{urban\_zone} field and the \texttt{TariffService} class directly implement this decree\allowbreak .
    
    \item \textbf{Office de la Topographie et de la Cartographie (OTC)}\\
    \url{http://www.otc.nat.tn/}\\
    National cadastral authority providing official boundary data and property identifiers. Future integration would sync declared property coordinates with OTC records to reduce fraud and verification overhead.
\end{enumerate}

\subsection{Backend Frameworks and Libraries}

\begin{enumerate}[label=\arabic*., resume]
    \item \textbf{Flask Documentation}\\
    \url{https://flask.palletsprojects.com/}\\
    Official documentation for Flask 3.0.2, the micro web framework powering TUNAX's backend. Consulted for application factory patterns, blueprint registration, and request lifecycle management.
    
    \item \textbf{Flask-SMOREST Documentation}\\
    \url{https://flask-smorest.readthedocs.io/}\\
    OpenAPI 3.0 REST framework built atop Flask. TUNAX uses Flask-SMOREST to auto-generate\allowbreak Swagger UI, validate request/response\allowbreak payloads via Marshmallow, and expose RESTful endpoints consistently\allowbreak .
    
    \item \textbf{SQLAlchemy Documentation}\\
    \url{https://docs.sqlalchemy.org/}\\
    Version 2.0.23 provides the ORM layer for TUNAX models. Consulted extensively for relationship patterns, polymorphic associations (e.g., \texttt{Tax} linking \texttt{Property} or \allowbreak \texttt{Land}), and declarative mapping conventions.
    
    \item \textbf{Alembic Documentation}\\
    \url{https://alembic.sqlalchemy.org/}\\
    Database migration framework. The \path{migrations/versions/} directory contains auto-generated scripts that evolve the schema.
    
    \item \textbf{Marshmallow Documentation}\\
    \url{https://marshmallow.readthedocs.io/}\\
    Version 3.24.1 powers the \texttt{schemas/} module for validation and serialization. Each API endpoint declares schemas enforcing data contracts and generating OpenAPI specifications.
    
    \item \textbf{Flask-JWT-Extended Documentation}\\
    \url{https://flask-jwt-extended.readthedocs.io/}\\
    JWT authentication extension (version 4.5.3) providing token issuance,\allowbreak refresh, and blacklist capabilities. TUNAX's \texttt{auth.py} resource and \texttt{role\_required} decorators depend on this library\allowbreak .
\end{enumerate}
\endgroup

\subsection{Security and Authentication}

\begingroup\raggedright
\setlength{\parfillskip}{0pt plus 1fil}
\begin{enumerate}[label=\arabic*., resume]
    \item \textbf{Werkzeug Security Utilities}\\
    \url{https://werkzeug.palletsprojects.com/en/stable/utils/}\\
    Password hashing via \texttt{pbkdf2:sha256} in the \texttt{User} model\allowbreak using \allowbreak \texttt{generate\_password\_hash} and \texttt{check\_password\_hash}\allowbreak .
    
    \item \textbf{Flask-Limiter Documentation}\\
    \url{https://flask-limiter.readthedocs.io/}\\
    Rate limiting extension that protects authentication endpoints (5 requests/minute) and general API routes (50 requests/hour) from abuse. TUNAX migrated to Redis-backed storage (January 2026) via \allowbreak \texttt{RATELIMIT\_STORAGE\_URI} for distributed rate limit enforcement across multiple backend workers in \allowbreak production deployments\allowbreak .
    
    \item \textbf{PyOTP (Python One-Time Password Library)}\\
    \url{https://pyauth.github.io/pyotp/}\\
    Implements TOTP-based two-factor authentication. TUNAX's \texttt{two\_factor.py} model and \allowbreak \path{/api/2fa/*}\allowbreak endpoints rely on PyOTP for code generation and verification\allowbreak .
    
    \item \textbf{RFC 7519: JSON Web Token (JWT)}\\
    \url{https://tools.ietf.org/html/rfc7519}\\
    Standardizes JWT structure and claims. TUNAX adheres to this specification when encoding \texttt{role} and \allowbreak \texttt{commune\_id} claims into access and refresh tokens.
\end{enumerate}
\endgroup

\subsection{Database Systems}

\begin{enumerate}[label=\arabic*., resume]
    \item \textbf{PostgreSQL 15 Documentation}\\
    \url{https://www.postgresql.org/docs/15/}\\
    Production-grade relational database. TUNAX uses PostgreSQL for JSON columns, CITEXT for case-insensitive email matching, and robust ACID guarantees in transaction-heavy workflows (e.g., payment processing).
    
    \item \textbf{Redis 7 Documentation}\\
    \url{https://redis.io/docs/}\\
    In-memory data store deployed as part of TUNAX's Docker Compose stack (port 6379). Serves dual purposes: distributed rate limit enforcement via Flask-Limiter and JWT token blacklist storage for secure logout functionality across multiple backend workers.
    
    \item \textbf{SQLite Documentation}\\
    \url{https://www.sqlite.org/docs.html}\\
    Lightweight file-based database used in development (\texttt{tunax.db}). Supports rapid schema iteration and zero-configuration deployment for local testing.
\end{enumerate}

\subsection{Geolocation and Geographic Services}

\begin{enumerate}[label=\arabic*., resume]
    \item \textbf{Nominatim (OpenStreetMap Geocoding)}\\
    \url{https://nominatim.org/release-docs/latest/}\\
    API endpoint: \url{https://nominatim.openstreetmap.org/}\\
    Converts declared property addresses to GPS coordinates (latitude, longitude). TUNAX's \texttt{geo.py} utility invokes Nominatim via the Geopy library, ensuring geospatial validation of TIB and TTNB declarations.
    
    \item \textbf{Geopy Documentation}\\
    \url{https://geopy.readthedocs.io/}\\
    Python library wrapping multiple geocoding services. TUNAX configures Geopy to query Nominatim while respecting rate limits and handling coordinate projection.
    
    \item \textbf{OpenStreetMap: Tunisia Administrative Boundaries}\\
    \url{https://www.openstreetmap.org/relation/192757}\\
    Community-maintained dataset of Tunisia's 264 communes. The \path{seed\_data/communes\_tn.csv} file derives from OpenStreetMap extracts and provides municipality names, postal codes, and geographic extents.
\end{enumerate}

\subsection{Containerization and Deployment}

\begin{enumerate}[label=\arabic*., resume]
    \item \textbf{Docker Documentation}\\
    \url{https://docs.docker.com/}\\
    Container platform documentation. TUNAX's \path{docker/Dockerfile} packages the Flask app, dependencies, and entrypoint into a portable image suitable for cloud or on-premise deployment.
    
    \item \textbf{Docker Compose Documentation}\\
    \url{https://docs.docker.com/compose/}\\
    Multi-container orchestration tool. The \path{docker-compose.yml} file defines services for the backend (Flask), database (PostgreSQL), and frontend (Nginx), enabling one-command environment provisioning.
    
    \item \textbf{Nginx Documentation}\\
    \url{https://nginx.org/en/docs/}\\
    Web server and reverse proxy. The \path{nginx.conf} file routes frontend static assets and proxies API requests to the Flask backend, decoupling client delivery from application logic.
\end{enumerate}

\subsection{API Standards and Specifications}

\begin{enumerate}[label=\arabic*., resume]
    \item \textbf{OpenAPI Specification 3.0}\\
    \url{https://swagger.io/specification/}\\
    Industry-standard API description format. Flask-SMOREST auto-generates an OpenAPI document at \texttt{/api/swagger.json}, which drives Swagger UI and client SDK generation.
    
    \item \textbf{RESTful API Design Best Practices}\\
    \url{https://restfulapi.net/}\\
    Resource-oriented architecture guide. TUNAX's endpoints follow REST principles: nouns for resources (\texttt{/properties}, \texttt{/lands}), HTTP verbs for actions (GET, POST, PUT, PATCH, DELETE), and HATEOAS links for discoverability.
\end{enumerate}

\subsection{Testing and Quality Assurance}

\begin{enumerate}[label=\arabic*., resume]
    \item \textbf{Insomnia REST Client}\\
    \url{https://insomnia.rest/}\\
    API testing tool. The \texttt{tests/insomnia\_collection.json} file contains pre-configured scenarios for authentication, property declaration, payment processing, and dispute workflows, enabling regression testing.
\end{enumerate}

\subsection{Observability and Monitoring (Future)}

\begin{enumerate}[label=\arabic*., resume]
    \item \textbf{Sentry Error Tracking}\\
    \url{https://docs.sentry.io/}\\
    Real-time error monitoring and alerting. Integration would capture uncaught exceptions, performance regressions, and user-facing errors in production.
    
    \item \textbf{Prometheus Monitoring}\\
    \url{https://prometheus.io/docs/}\\
    Time-series metrics collection. Future instrumentation could expose Flask endpoint latencies, database query durations, and custom business metrics (e.g., declarations per day).
    
    \item \textbf{Grafana Dashboards}\\
    \url{https://grafana.com/docs/}\\
    Visualization layer for Prometheus metrics. Municipal admins could monitor system health, user activity, and revenue trends via real-time dashboards.
\end{enumerate}

\subsection{Additional Libraries and Utilities}

\begin{enumerate}[label=\arabic*., resume]
    \item \textbf{Flask-CORS Documentation}\\
    \url{https://flask-cors.readthedocs.io/}\\
    Cross-Origin Resource Sharing middleware. Enables frontend (served via Nginx on port 80) to communicate with backend (Flask on port 5000) without browser security errors.
    
    \item \textbf{python-dotenv Documentation}\\
    \url{https://pypi.org/project/python-dotenv/}\\
    Loads environment variables from \texttt{.env} files. TUNAX uses dotenv to configure database URLs, JWT secrets, and rate limits without hardcoding sensitive values.
    
    \item \textbf{PyYAML Documentation}\\
    \url{https://pyyaml.org/wiki/PyYAMLDocumentation}\\
    YAML parser for Python. The \texttt{tariffs\_2025.yaml} file defines service-based rates, exemption thresholds, and penalties in a human-readable format loaded at runtime.
\end{enumerate}

\subsection{Payment Gateways (Planned Integration)}

\begin{enumerate}[label=\arabic*., resume]
    \item \textbf{Clictopay (Tunisia)}\\
    \url{https://www.clictopay.com.tn/}\\
    Tunisian online payment gateway supporting credit/debit cards and e-dinars. Future integration would enable citizens to settle tax bills directly through the TUNAX frontend.
    
    \item \textbf{SMT (Soci\'et\'e Mon\'etique Tunisie)}\\
    \url{https://www.smt.tn/}\\
    National payment card network. Integration would allow municipal finance offices to reconcile electronic payments with bank transfers and card settlements.
\end{enumerate}

\subsection{Version Control and Collaboration}

\begin{enumerate}[label=\arabic*., resume]
    \item \textbf{Git Documentation}\\
    \url{https://git-scm.com/doc}\\
    Distributed version control system. TUNAX's repository uses Git for branching, merging, and audit trails. The commit history documents architectural evolution and AI-assisted contributions.
    
    \item \textbf{GitHub Copilot}\\
    \url{https://github.com/features/copilot}\\
    AI-powered code completion assistant. Copilot accelerated blueprint creation, refactoring, and documentation drafting while maintainers retained oversight through code reviews and test validation.
\end{enumerate}

\section{Conclusion}
This academic report documented the technical and contextual dimensions of TUNAX, a comprehensive Tunisian municipal tax management platform. By uniting legal requirements, robust architecture, and user-centric dashboards, the system demonstrates how software engineering can modernize public finance. The platform successfully operationalizes the 2025 Code de la Fiscalit\'e Locale through Marshmallow schemas, SQLAlchemy models, and Flask-Smorest blueprints, providing a reproducible reference implementation for municipalities nationwide. Continued collaboration with municipalities, security audits, and user research will ensure the platform scales sustainably across Tunisia's diverse regions.

\end{document}
